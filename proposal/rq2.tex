 \textbf{Motivation.}
In RQ1, we find that our predictor performs well in prediction of the existence of performance regression. If we can predict the magnitude of these performance regressions, we may provide guidance or automated tooling support for developer to prevent the large effect on performance regression.

\textbf{Approach.}
In RQ1, we find that there exist prevalent performance regressions that are introduced by code changes. To address RQ2, we build an ordinal prediction model for the magnitude of performance regression introducing change based on the measures (see Table~\ref{tab:measures}) and effect size of the regression. we obtain the results of effect size (large, medium, small regression and not significant difference) for every test case in our subject systems. We also consider four types of performance counters.

We follow two steps in our approach to predicting the magnitude of performance regression. First, we use \emph{Weka} to build the ordinal model OrdinalClassClassifier~\cite{Frank2001}, which allows standard classification algorithms to be applied to ordinal class problems to predict the effect size of regression in four performance metrics, \emph{Runtime, CPU, Memory and IO}. Second, we compare the accuracy between different metrics and compare the efficiency between ordinal models with the logistic regression model built in RQ1. 

To validate how well the model predicts the magnitude of performance regression introducing changes, we also use two metrics, \emph{precision} and \emph{recall}. And perform 10-fold cross validation to verify the stability of the prediction.

\textbf{Results.} We find that there exist more performance-regression-prone tests with large effect sizes than medium and small (see Table~\ref{tab:effect}). Such results imply that developers may not ignore these performance regressions since they may have a large impact on system performance.

\textbf{Our predictor achieves an average precision of 54\% percent and recall of 66\% percent.} We employ our magnitude prediction model into these two systems and the result is shown in Table~\ref{tab:ordinal}. We can find that the average precision of 59\% percent in \emph{Hadoop} is higher than the average precision of 48\% percent in \emph{Rxjava}. The average precision of 60\% in performance counter \emph{Runtime} is higher than other three performance counters (around 50\%).  We infer that the correlation is low between the effect sizes calculated with runtime and the physical counters. On the other hand, the effect sizes calculated with the physical counters may have medium or large correlations with each other. We believe that these correlations are often due to the nature of the software itself (e.g., database accessing software may have its CPU usage highly correlated with I/O).

\begin{table}[]
	\centering
	\footnotesize
	\caption{Precision and recall of predicting the magnitude of performance regression.}
	\label{tab:ordinal}
\begin{tabular}{|c|r|r|r|r|r|r|r|r|}
	\hline
	\multirow{2}{*}{}             & \multicolumn{2}{c|}{Runtime(\%)}                      & \multicolumn{2}{c|}{CPU(\%)}                          & \multicolumn{2}{c|}{Memory(\%)}                       & \multicolumn{2}{c|}{IO(\%)}                           \\ \cline{2-9} 
	& \multicolumn{1}{c|}{pre.} & \multicolumn{1}{c|}{rec.} & \multicolumn{1}{c|}{pre.} & \multicolumn{1}{c|}{rec.} & \multicolumn{1}{c|}{pre.} & \multicolumn{1}{c|}{rec.} & \multicolumn{1}{c|}{pre.} & \multicolumn{1}{c|}{rec.} \\ \hline
	Hadoop                        & 67.5                      & 73.5                      & 48.8                      & 54.1                      & 54.7                      & 64.9                      & 65.7                      & 72.8                      \\ \hline
	Rxjava                        & 52.2                      & 64.2                      & 45.7                      & 67.6                      & 45.9                      & 61.5                      & 48.5                      & 69.7                      \\ \hline
	\multicolumn{1}{|l|}{Average} & 59.9                      & 68.9                      & 47.3                      & 60.8                      & 50.3                      & 63.2                      & 57.1                      & 71.3                      \\ \hline
\end{tabular}
\end{table}

\textbf{Our predictor performs better in the large effect than the medium effect on performance regression.} 189 tests are large effect and 149 tests are medium effect on regression in \emph{Hadoop}. And 1745 benchmarks are large effect and 1355 benchmarks are medium effect on regression in \emph{Rxjava} (see Table~\ref{tab:effect}). The average precision of 56\% percent in large effect is higher than the average precision of 47\% percent in medium effect on regression (see Table~\ref{tab:preeffect}). It reflects that the change causing regression with large effect size carries more obvious change features. It is super meaningful because most developers focus more on the performance regression with a large effect because in many cases, performance regression with large effect can cause large effect on users. Similarly, the precision of 69.5\% in \emph{Runtime} is higher than other performance counters in different effect sizes.

\begin{table}[]
	\centering
	\footnotesize
	\caption{Precision of predicting large and medium performance regression.}
	\label{tab:preeffect}
\begin{tabular}{|c|r|r|r|r|r|r|r|r|}
	\hline
	& \multicolumn{2}{c|}{Runtime(\%)}                      & \multicolumn{2}{c|}{CPU(\%)}                          & \multicolumn{2}{c|}{Memory(\%)}                       & \multicolumn{2}{c|}{IO(\%)}                           \\ \hline
	& \multicolumn{1}{c|}{large} & \multicolumn{1}{c|}{med} & \multicolumn{1}{c|}{large} & \multicolumn{1}{c|}{med} & \multicolumn{1}{c|}{large} & \multicolumn{1}{c|}{med} & \multicolumn{1}{c|}{large} & \multicolumn{1}{c|}{med} \\ \hline
	Hadoop                        & 65.6                       & 34.2                     & 55.3                       & 45.4                     & 41.7                       & 48.5                     & 46.8                       & 43.4                     \\ \hline
	Rxjava                        & 73.3                       & 51.6                     & 67.6                       & 52.4                     & 65.8                       & 49.2                     & 32.1                       & 39.6                     \\ \hline
	\multicolumn{1}{|l|}{Average} & 69.5                       & 47.9                     & 61.5                       & 48.9                     & 53.8                       & 48.8                     & 39.5                       & 41.5                     \\ \hline
\end{tabular}
\end{table}

\textbf{Magnitude predictor performs worse than the existence predictor.} The average precision of 70.3\% percent in the existence predictor is higner than the average precision of 54\% percent in the ordinal magnitude predictor. We infer the reason is that the two kinds of models train on the same dataset (change measure is the same). However, in the ordinal predictor, it needs to recognize five classes rather two classes in the logistic regression model. To improve the accuracy of the ordinal model, we should add more change measures based on the measures in the logistic model.

\fbox{\parbox{23.5em}{ \emph{ Our predictor achieves an average precision of 54\% percent and recall of 66\% percent. We find that there exist more performance-regression-prone tests with large effect sizes than medium. Our predictor performs better in the large effect than medium effect on performance regression.}}}
