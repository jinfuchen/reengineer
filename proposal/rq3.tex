
%\noindent \textbf{Motivation}

%In RQ1, we find that there exist prevalent performance regressions that are introduced by code changes. If we can understand what cause the introduction of these performance regressions, we may provide guidance or automated tooling support for developer to prevent the regressions during code change. 

\noindent \textbf{Data and Approach}
To address RQ3, we analyze and compare the regression coefficients of the logistic models from RQ1 and RQ2. To measure the effect of every change metric, we keep all of the metrics at their original value, except for the metric whose effect we wish to measure. We increase the value of that metric by 10\% off the original value and re-calculate the predicted probability. We then calculate the percentage of difference caused by increasing the value of that metric by 10\%. The effect of a metric can be positive or negative. A positive effect means that a higher value of the factor increases the likelihood, whereas a negative effect means that a higher value of the factor decreases the likelihood of the dependent variable.

\noindent \textbf{Results.}





