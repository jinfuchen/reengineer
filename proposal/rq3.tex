
%\noindent \textbf{Motivation}

%In RQ1, we find that there exist prevalent performance regressions that are introduced by code changes. If we can understand what cause the introduction of these performance regressions, we may provide guidance or automated tooling support for developer to prevent the regressions during code change. 

\textbf{Data and Approach.}
Each change measure has different impact on the software quality. Finding the more important change measure can guide delevopers and maintainers to be more concerned of these measures. To address RQ3, we analyze and compare the regression coefficients of the logistic models from RQ1 and RQ2. 

To measure the effect of every change metric, we keep all of the metrics at their original value, except for the metric whose effect we wish to measure. We increase the value of that metric by 10\% off the original value and re-calculate the predicted probability. We then calculate the percentage of difference caused by increasing the value of that metric by 10\%. The effect of a metric can be positive or negative. A positive effect means that a higher value of the factor increases the likelihood, whereas a negative effect means that a higher value of the factor decreases the likelihood of the dependent variable.

\textbf{Results.} \textbf{Change measure in the dimension of \emph{size} are more important than other dimension of change measures in the commit-level metrics. LA are more important change measure in the dimension of \emph{size}.}

\textbf{CL is more important other change measures in the dimension of performance-relevant meteics.}





