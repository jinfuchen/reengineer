Prior study of JIT defect prediction focus more on the risk of commit based on bug rather than performance regressions. In our project, we conduct an empirical study on the JIT prediction of performance regression introducing code changes  in two open source software \emph{Hadoop} and \emph{RxJava}. To build a change risk model to predict JIT performance regression, we combine the basic commit-level measures, such as the number of modified subsystems and the purpose of the code change, with the performance-revelant measures we added, such as changing conditions and introducing synchronization. In particular, this paper makes the following contributions:
\vspace{-0.2cm}
\begin{itemize} \itemsep 0em
\item To the best of our knowledge, our work is the first to predict performance regressions at the commit level. 
\item We propose a statistically rigorous approach to identifying performance regression introducing code changes. Further research can adopt our methodology in studying performance regressions.
\item We find that change measures in dimension \emph{Size} are more risk-inceasing factors. Change measures \emph{Loop} and \emph{Synchronization} are more likely to cause performance regression.
\end{itemize}
